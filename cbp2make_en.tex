\section{Cbp2make Utility}\label{sec:cbp2make}

A Makefile generation tool for \codeblocks IDE by Mirai Computing. The text of this section comes from his cbp2make Wiki on SourceForge.
\hint{Cbp2make is not a \codeblocks plugin, but a full console application, located in the main \codeblocks directory, to generate makefile(s) from the internal \codeblocks generating system}

\subsection{About}

"cbp2make" is a stand-alone build tool that allows you to generate makefile(s) for GNU Make out of \codeblocks IDE project or workspace file. (See also \url{https://forums.codeblocks.org/index.php/topic,13675.0.html]})

\subsection{Usage}
\genterm{Create makefile for a single project or workspace}

Let's assume you have a project "my\_project.cbp" and you need a makefile for this project. In this simplest case all you have to do is:
\begin{lstlisting}
cbp2make -in my_project.cbp
\end{lstlisting}

Same thing applies to workspaces.
\begin{lstlisting}
cbp2make -in my_projects.workspace
\end{lstlisting}

\genterm{Create makefile with another filename}
By default, "cbp2make" will append ".mak" extension to the project name to compose a filename for makefile.
If you want to change that, use following command:

\begin{lstlisting}
cbp2make -in my_project.cbp -out Makefile
\end{lstlisting}

\genterm{Create makefile for another platform}
If you are working in GNU/Linux and you want to generate a makefile for Windows or the other way around, you can specify one or more platforms for which you need makefiles.

\begin{lstlisting}
cbp2make -in my_project.cbp -windows
cbp2make -in my_project.cbp -unix
cbp2make -in my_project.cbp -unix -windows -mac
cbp2make -in my_project.cbp --all-os
\end{lstlisting}
"cbp2make" will append ".unix" or ".windows" or ".mac" suffix to makefile name for each platform respectively.

\genterm{Create makefile for multiple projects or workspaces}
If you have more than one independent project or workspace, you can process them at once, but you will need a text file containing the list of projects, e.g., "projects.lst", with one project filename per line.

\begin{lstlisting}
    my_project.cbp
    my_other_project.cbp 
\end{lstlisting}

And then you can process them using command:
\begin{lstlisting}
cbp2make -list -in projects.lst
\end{lstlisting}

\subsection{Configuration}

Some installation-specific or project-specific options, primarily toolchain settings, can be saved to a configuration file. By default (\textit{since rev.110}), cbp2make does not save any settings to a configuration file unless the user explicitly specifies the \codeline{"--config"} option. A configuration file can be either global (stored in user profile / home directory) or local (stored in current directory).

Please, keep in mind that since cbp2make is in early stage of development, an old configuration file may become incompatible with new tool version and it may be necessary to update it manually or initialize a new one.

\genterm{Initialization}

\begin{lstlisting}
cbp2make --config options --global
cbp2make --config options --local
\end{lstlisting}

\genterm{Later use}

When cbp2make is invoked, first it tries to load a local configuration file. If a local configuration is missing, next attempt will be to load a global one. If this attempt is not successful either, the default built-in configuration is used. Configuration lookup order can be overridden with \codeline{"--local"} or \codeline{"--global"} command line options. If one of options is supplied to cbp2make, non-specified configuration is not picked up even if the specified one is missing and non-specified do exists.

\genterm{Default lookup order}

\begin{lstlisting}
cbp2make -in project.cbp -out Makefile}
\end{lstlisting}

\genterm{Explicitly specified configuration}

\begin{lstlisting}
cbp2make --local -in project.cbp -out Makefile
cbp2make --global -in project.cbp -out Makefile
\end{lstlisting}

\subsection{Command line syntax}

Generate makefile:
\begin{verbatim}
cbp2make -in <project_file> [-cfg <configuration>] [-out <makefile>]
[-unix] [-windows] [-mac] [--all-os] [-targets "<target1>[,<target2>[, ...]]"]
[--flat-objects] [--flat-objpath] [--wrap-objects] [--wrap-options]
[--with-deps] [--keep-objdir] [--keep-outdir] [--target-case keep|lower|upper]

cbp2make -list -in <project_file_list> [-cfg <configuration>]
[-unix] [-windows] [-mac] [--all-os] [-targets "<target1>[,<target2>[, ...]]"]
[--flat-objects] [--flat-objpath] [--wrap-objects] [--wrap-options]
[--with-deps] [--keep-objdir] [--keep-outdir] [--target-case keep|lower|upper]
\end{verbatim}

Manage toolchains:
\begin{verbatim}
cbp2make --config toolchain --add \[-unix|-windows|-mac\] -chain <toolchain>
cbp2make --config toolchain --remove \[-unix|-windows|-mac\] -chain <toolchain>
\end{verbatim}

Manage build tools:
\begin{verbatim}
cbp2make --config tool --add \[-unix|-windows|-mac\] -chain <toolchain>
         -tool <tool> -type <type> <tool options>
         
cbp2make --config tool --remove \[-unix|-windows|-mac\] -chain <toolchain>
         -tool <tool>
\end{verbatim}

Tool types:      
\begin{verbatim}
    pp=preprocessor as=assembler cc=compiler rc=resource compiler
    sl=static linker dl=dynamic linker el=executable linker
    nl=native linker
\end{verbatim}

Tool options (common):
\begin{verbatim}
    -desc <description> -program <executable> -command <command_template>
    -mkv <make_variable> -srcext <source_extensions> -outext <output_extension>
    -quotepath <yes|no> -fullpath <yes|no> -unixpath <yes|no>
\end{verbatim}

Tool options (compiler):
\begin{verbatim}
    -incsw <include_switch> -defsw <define_switch> -deps <yes|no>
\end{lstlisting}

Tool options (linker):
\begin{verbatim}
    -ldsw <library_dir_switch> -llsw <link_library_switch> -lpfx <library_prefix>
    -lext <library_extension> -objext <object_extension> -lflat <yes|no>
\end{verbatim}

Manage platforms:
\begin{verbatim}
cbp2make --config platform \[-unix|-windows|-mac\] \[-pwd <print_dir_command>\]
         \[-cd <change_dir_command>\] \[-rm <remove_file_command>\]
         \[-rmf <remove_file_forced>\] \[-rmd <remove_dir_command>\]
         \[-cp <copy_file_command>\] \[-mv <move_file_command>\]
         \[-md <make_dir_command>\] \[-mdf <make_dir_forced>\]
         \[-make <default_make_tool>\]         
\end{verbatim}

\begin{samepage}
Manage global compiler variables:
\begin{verbatim}
cbp2make --config variable --add \[-set <set_name>\] -name <var_name>
        \[-desc <description>\] \[-field <field_name>\] -value <var_value>
        
cbp2make --config variable --remove \[-set <set_name>\] \[-name <var_name>\]
        \[-field <field_name>\]
\end{verbatim}
\end{samepage}

Manage options:
\begin{verbatim}
cbp2make --config options --default-options "<options>"    
cbp2make --config show
\end{verbatim}

Common options:
\begin{verbatim}
cbp2make --local         // use configuration from current directory
cbp2make --global        // use configuration from home directory
cbp2make --verbose       // show project information
cbp2make --quiet         // hide all messages
cbp2make --help          // display this message
cbp2make --version       // display version information
\end{verbatim}

\genterm{Options}

\begin{lstlisting}
"Makefile generation"

    -in <project_file>   // specifies an input file or a list of files;

    -cfg <configuration> // specifies a configuration file, see also "--local"
                            and "--global" options;

    -out <makefile>      // specifies the name of a makefile or a list of
                            makefiles;

    -unix                // enables Unix / Linux compatible makefile generation;

    -windows             // enables Windows compatible makefile generation;

    -mac                 // enables Macintosh compatible makefile generation;

    --all-os             // enables all target platforms at once;

    -targets "<target1>[,<target2>[, ...]]" // specifies the only build targets
                                               that a makefile will be made for;

    --flat-objects       // forces "flat" names for object files with limited
                            character set;

    --flat-objpath       // forces "flat" paths for object files with no
                            subdirectories;

    --wrap-objects       // allows to use multiline lists of objects which
                            makes a makefile easier to read;

    --wrap-options       // allows to use multiline macros;

    --with-deps          // allows a built-in dependency scanner for C/C++
                            projects;

    --keep-objdir        // disables command that erase directories for
                            object files in 'clean' target;

    --keep-outdir        // disables command that erase directory for an
                            output binary file in 'clean' target;

    --target-case [keep|lower|upper] // specifies style for makefile targets;
\end{lstlisting}