\section{Extension Editor Tweaks}\label{sec:editor_tweaks}

Le plugin EditorTweaks (modifications d'édition) apporte plusieurs fonctionnalités différentes. Sur une base de travail fichier à fichier, il contrôle :

\begin{itemize}[noitemsep]
\item le repliement de mots ;
\item la numérotation des lignes ;
\item l'interprétation de la touche tab (caractère de tabulation ou espaces) ;
\item le nombre de caractères espace remplaçant la touche tab ;
\item les caractères de fin de ligne (carriage-return + linefeed; carriage-return; linefeed) ;
\item la visualisation des caractères de fin de ligne ;
\item sur demande, la suppression des espaces blancs en fin de ligne ;
\item sur demande, la synchronisation des caractères de fin de ligne ;
\item la suppression de la touche d'insertion.
\end{itemize}

Depuis la fusion avec le plugin "Aligner", il peut rendre des sections de code plus lisibles en les alignant sur un caractère spécifique.\newline
Par exemple, aligner sur le caractère "=" dans :

\begin{lstlisting}
int var = 1;
int longVarName = 2;
int foobar = 3;
\end{lstlisting}

se traduira par :

\begin{lstlisting}
int var         = 1;
int longVarName = 2;
int foobar      = 3;
\end{lstlisting}
