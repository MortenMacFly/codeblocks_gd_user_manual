\section{Extension Variables d'Environnement}\label{sec:EnvVar_Plugin}

D'après le wiki de \codeblocks. Voir aussi la \pxref{sec:EnvVars_Cfg}.

L'extension \textbf{Éditeur de variables d'environnement} permet de définir des variables d'environnement du système dans le cadre de \codeblocks.\newline
L'utilisateur peut avoir plusieurs ensembles qui contiennent 1..n variables d'environnement.\newline
L'utilisateur peut passer d'un ensemble à l'autre via la boîte de dialogue de configuration des variables d'environnement.\newline
En outre, l'extension EnvVars apporte une option aux projets (dans la configuration du projet) pour appliquer un ensemble EnvVar particulier à activer (et à utiliser pendant la compilation).

La boîte de dialogue permettant de modifier les ensembles se trouve dans \menu{Paramètres,Environnement,Variables d'environnement}.\newline
La boîte de dialogue permettant de choisir l'ensemble actif pour le projet en cours se trouve dans \menu{Projet,Propriétés,Options EnvVar}.\newline

\textbf{Script binding}

Cette extension apporte sa fonctionnalité via un "squirrel binding" : 

{\footnotesize
\begin{longtable}{|l|l|l|l|}\hline
\textbf{Valeur de retour}&\textbf{Nom}      &\textbf{Arguments}     &\textbf{Remarques}                     \\ \hline
\endhead    % Pour répéter la ligne de titre si besoin
wxArrayString   &EnvvarGetEnvvarSetNames    &                       &Retourne tous les ensembles            \\
                &                           &                       &envvar disponibles                     \\ \hline
wxString        &EnvvarGetActiveSetName     &                       &Retourne le nom de l'ensemble          \\
                &                           &                       &actif courant                          \\
                &                           &                       &(depuis config, /active\_set)          \\ \hline
wxArrayString   &EnvVarGetEnvvarsBySetPath  &const wxString         &Retourne les envvars d'un              \\
                &                           &set\_name              &chemin d'ensembles envvars             \\
                &                           &                       &dans la config                         \\ \hline
bool            &EnvvarSetExists            &const wxString         &Vérifie si un ensemble d'envvars       \\
                &                           &set\_name              &existe effectivement dans la config    \\ \hline
bool            &EnvvarSetApply             &const wxString\&       &Applique un ensemble envvar            \\
                &                           &set\_name,             &spécifique de la config                \\
                &                           &bool even\_if\_active  &(sans interaction de l'IU)             \\ \hline
void            &EnvvarSetDiscard           &const wxString         &Ignore un ensemble envvar              \\
                &                           &                       &spécifique de la config                \\
                &                           &                       &(sans interaction de l'IU)             \\ \hline
bool            &EnvvarApply                &const wxString key,    &Applique un envvar spécifique          \\
                &                           &const wxString value   &                                       \\ \hline
bool            &EnvvarDiscard              &const wxString key     &Ignore un envvar                       \\ \hline
\caption{Squirrel binding}
\end{longtable}
\par}

\textbf{NOTE} : Les arguments "value" sont automatiquement générés à partir des macros. Vous n'avez pas besoin d'appeler ReplaceMacros() sur ceux-ci.

Beaucoup d'autres fonctions de script sont disponibles. Regardez dans \url{https://wiki.codeblocks.org/index.php/Scripting_commands}

\textbf{Exemple}

Dans les fenêtres des étapes de post ou pré-génération :
\begin{lstlisting}
[[EnvvarApply(_("test"),_("testValue"));]]
echo %test%
\end{lstlisting}

