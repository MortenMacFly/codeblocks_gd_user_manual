\section{File formats description}\label{sec:file_formats}

Partialy extracted from the Wiki.

\codeblocks projects/workspaces are described in XML files. Here is contained a short description for each one of them.

This information is of interest to anyone wanting to write an importer/exporter/generator for other build systems/environments and therefore add support for \codeblocks.

\begin{description}
\item[Workspace file] (*.workspace) Defines a \codeblocks workspace (collection of projects). See details below (\pxref{sec:workspace_file}) or \url{https://wiki.codeblocks.org/index.php/Workspace_file}.
\item[Project file] (*.cbp) Defines a \codeblocks project. See details on \url{https://wiki.codeblocks.org/index.php/Project_file}.
\end{description}

Additional files used as of December 12, 2012 (from the merge of the XML compiler branch):

\begin{description}
\item[Compiler file] (compiler\_*.xml) Defines a \codeblocks compiler interface and auto-detection routines. See details on \url{https://wiki.codeblocks.org/index.php/Compiler_file}.
\item[Compiler options file] (options\_*.xml) Defines the options and regular expressions for a \codeblocks compiler. See details on \url{https://wiki.codeblocks.org/index.php/Compiler_options_file}.
\end{description}

\textit{\codeblocks also produces a couple of other files (*.layout and *.depend) but they only contain state information so they are not really useful to anyone else other than \codeblocks itself.}

\subsection*{Workspace file}\label{sec:workspace_file}

The workspace XML file is a very simple one.

A workspace is a collection of projects. Essentially the workspace file does exactly that: describes a collection of projects. But let's see the contents of a sample workspace:

\begin{lstlisting}[language=XML]
<?xml version="1.0" encoding="UTF-8" standalone="yes" ?>
<CodeBlocks_workspace_file>
	<Workspace title="Test Workspace">
		<Project filename="TestConsole/TestConsole.cbp" active="1">
			<Depends filename="TestLib/TestLib.cbp" />
		</Project>
		<Project filename="TestLib/TestLib.cbp" />
	</Workspace>
</CodeBlocks_workspace_file>
\end{lstlisting}

This XML text defines the workspace named "Test Workspace" containing two projects:

\begin{itemize}
\item TestConsole/TestConsole.cbp and
\item TestLib/TestLib.cbp
\end{itemize}

Additionally, it defines a project dependency: the TestConsole project \textit{depends} on the TestLib project. This instructs \codeblocks to always make sure that the TestLib project is built \textit{before} the TestConsole project.

\textbf{NOTE}: \textit{This is a build-order dependency that is being set. This will \_not\_ force a re-link of the TestConsole output (which is an executable) when the library generated by TestLib is updated. This behavior is influenced by another setting in the project file. See the Project file description for this.}

Well, I 'd love to add various comments in the XML itself, to describe each element, but as you see it's pretty simple and straightforward :). The only thing that requires perhaps some explanation is the "active" attribute seen in the "Project" element for the TestConsole project. That attribute appears only when its value equals to "1" and in only one "Project" element inside the workspace. All it does is define which of the workspace projects will be the active one by default, when opening the workspace in \codeblocks.


That's all. 