\section{NassiShneiderman plugin}\label{sec:nassishneiderman}

NassiShneiderman plugin allows the creation of Nassi Shneiderman diagrams within \codeblocks (\cite{url:nassi}). 

\subsection{Create a diagram}

There are two possibilities to create a diagram.

\begin{enumerate}
\item To create an empty diagram select the menu options \menu{File,New,Nassi Shneiderman diagram}.
\item The second option is to creates a diagram from C/C++ source code. 
\end{enumerate}

In an editor window highlight some code to create the diagram from. For example the body of a function/method from the opening braces till the closing brace. Then right click the selection and choose \menu{Nassi Shneiderman,Create diagram} (see \pxref{fig:NassiShneidermanCreate1}). 

\screenshot{NassiShneidermanCreate1}{NassiShneiderman Create}

You should get a new diagram (see \pxref{fig:NassiShneidermanCreate2}).

\screenshot{NassiShneidermanCreate2}{NassiShneiderman Diagram Example}

The parser has some limitations:

\begin{itemize}
\item Comments can not be at the end of a branch.
\item From a definition of a function it is only able to parse the body, not the declaration.
\item For sure, you will find a lot more... 
\end{itemize}

\subsection{Editing structograms}
\subsubsection{What to do with a diagram?}

You can do a lot of things with a structogram:

\begin{enumerate}
\item Store for later usage. The diagram can be stored \menu{File,Save file} or \menu{File,Save file as...}.
\item It is possible to export to different formats \menu{File,Export}
    \begin{itemize}
    \item "Export source..." to save as C source file.
    \item "StrukTeX" to use in your documentation in LaTeX.
    \item "PNG" or "PS" and eventually "SVG" to have a diagram in an image format known by a lot of other tools.
    \end{itemize}        
\item Directly insert the code into the editor: Open or create a diagram. Back in an editor window right click and choose \menu{Nassi Shneiderman,insert from xy} (You get a list with all open diagrams here).
\item Drag'n'Drop the diagram (or parts of it) to other tools. For example to OpenOffice Writer to get an image in your documentation.
\end{enumerate}

If the chosen diagram has a selection, the export or code-generation takes only this part of the diagram. 

\subsubsection{Extensions}

The NassiShneiderman plugin supports some extensions of Nassi-Shneiderman diagrams: 

\begin{itemize}
\item break has a special brick with an "right-arrow"
\item continue has a special brick with a "left-arrow"
\item To be able to create diagrams from c/c++ switch statements, the selection does not implicitly break before a case. The different cases are vertically aligned. Supports C and C++.
\end{itemize}

