\section{Exporter du code Source}\label{sec:src_exporter}

Il est souvent nécessaire de transférer du code source vers d'autres applications ou vers des e-mails. Si le texte est simplement copié, le formatage est perdu, ce qui rend le texte peu clair.
La fonction exporter de \codeblocks est une des solutions dans ce type de situations. Le format requis pour le fichier exporté peut être sélectionné via \menu{Fichier,Exporter}. Le programme adoptera alors le nom de fichier et le répertoire cible en fonction du fichier source ouvert et les proposera pour enregistrer le fichier à exporter. L'extension de fichier appropriée à chaque cas de figure sera déterminée par le type de l'exportation. Les formats suivants sont disponibles :

\begin{description}
\item[html] Un format de type texte qui peut être affiché dans un navigateur web ou dans un traitement de texte.
\item[rtf] Le format Rich Text qui est un format basé sur du texte et qui peut être ouvert dans un traitement de texte comme Word ou OpenOffice.
\item[odt] Le format Open Document Text qui est un format standardisé spécifié par Sun et O'Reilly. Ce format peut être traité par Word, OpenOffice et d'autres traitements de texte.
\item[pdf] Le format Portable Document qui peut être ouvert par des applications comme Acrobat Reader.
\end{description}

