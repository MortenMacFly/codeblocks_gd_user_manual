\section{Extension NassiShneiderman}\label{sec:nassishneiderman}

L'extension NassiShneiderman permet de créer des diagrammes de Nassi Shneiderman depuis \codeblocks (\cite{url:nassi}). 

\subsection{Création d'un diagramme}

Vous avez deux possibilités pour créer un diagramme.

\begin{enumerate}
\item Pour créer un diagramme vide, sélectionnez les options de menu \menu{Fichier,Nouveau,Diagramme de Nassi Shneiderman}.
\item La deuxième option consiste à créer un diagramme depuis le code source C/C++. 
\end{enumerate}

Dans une fenêtre de l'éditeur, sélectionnez une partie de code pour en créer un diagramme. Par exemple le corps d'une fonction/méthode depuis l'accolade ouvrante jusqu'à l'accolade fermante. Puis, via un clic droit sur la sélection, choisissez \menu{Nassi Shneiderman,Créer un diagramme} (voir \pxref{fig:NassiShneidermanCreate1}). 

\screenshot{NassiShneidermanCreate1}{NassiShneiderman Création}

Vous devriez obtenir un nouveau diagramme (voir \pxref{fig:NassiShneidermanCreate2}).

\screenshot{NassiShneidermanCreate2}{NassiShneiderman Exemple de Diagramme}

L'analyseur a quelques limitations:

\begin{itemize}
\item Des commentaires ne peuvent pas être placés en fin de branche.
\item Depuis la définition d'une fonction, on ne peut analyser que le corps de la fonction, pas la déclaration.
\item Bien sûr, vous en trouverez bien d'autres... 
\end{itemize}

\subsection{Édition de structogrammes}
\subsubsection{Que faire avec un diagramme ?}

Vous pouvez faire plein de choses avec un structogramme :

\begin{enumerate}
\item L'enregistrer pour l'utiliser plus tard. On peut l'enregistrer via \menu{Fichier,Enregistrer le fichier} ou \menu{Fichier,Enregistrer le fichier sous...}.
\item On peut l'exporter dans différents formats \menu{Fichier,Exporter}
    \begin{itemize}
    \item "Exporter la source..." pour l'enregistrer comme fichier source en C.
    \item "StrukTeX" pour l'utiliser dans une documentation sous LaTeX.
    \item "PNG" ou "PS" et éventuellement "SVG" pour obtenir le diagramme dans un format image connu de nombreux autres outils.
    \end{itemize}        
\item Insérer directement le code dans l'éditeur : Ouvrir ou créer un diagramme. De retour dans la fenêtre d'édition, faites un clic droit et choisissez \menu{Nassi Shneiderman,insérer en xy} (Vous obtenez ici une liste de tous les diagrammes ouverts).
\item Glisser/Déposer le diagramme (ou une partie) dans d'autres outils. Par exemple vers OpenOffice Writer afin d'y insérer une image dans votre documentation.
\end{enumerate}

Si le diagramme choisi comporte une sélection, l'exportation ou la génération de code ne portera que sur cette partie de diagramme. 

\subsubsection{Extensions}

L'extension NassiShneiderman supporte quelques extensions des diagrammes de Nassi-Shneiderman : 

\begin{itemize}
\item séparation d'une brique spécifique avec la "flèche droite"
\item continuer sur une brique spécifique avec la "flèche gauche"
\item Pour être en mesure de créer des diagrammes avec des instructions c/c++ "switch", la sélection ne doit pas être implicitement interrompue avant un "case". Les différents "cases" sont alignés verticalement. Support de C et C++.
\end{itemize}

