\subsection{Éditeur}

{\small
\begin{longtable}{|l|l|}\hline
\textbf{Fonction}		        &	\textbf{Raccourci clavier}  \\ \hline
\endhead    % Pour répéter la ligne de titre si besoin
Défaire la dernière action 	    &	Ctrl+Z                      \\ \hline
Refaire la dernière action 	    &	Ctrl+Shift+Z                \\ \hline
Couper le texte sélectionné     &   Ctrl+X                      \\ \hline
Copier le texte sélectionné     &   Ctrl+C                      \\ \hline
Coller le texte                 &   Ctrl+V                      \\ \hline
Sélectionner tout le texte      &   Ctrl+A                      \\ \hline
Permuter en-têtes / source 	    &	F11                         \\ \hline
Commenter le code surligné      &	Ctrl+Shift+C                \\ \hline
Décommenter le code surligné    & 	Ctrl+Shift+X                \\ \hline
Dupliquer la ligne où est le curseur      & 	Ctrl+D          \\ \hline
Auto-complète / Abréviations    & 	Ctrl+Space/Ctrl+J           \\ \hline
Afficher les astuces            &	Ctrl+Shift+Space            \\ \hline
Permuter la ligne où est le curseur avec celle au-dessus    &	Ctrl+T\\ \hline
Bascule la marque 	            &	Ctrl+B                      \\ \hline
Aller à la marque précédente 	&	Alt+PgUp                    \\ \hline
Aller à la marque suivante  	&	Alt+PgDown                  \\ \hline
Changer le repliement de bloc 	&	F12                         \\ \hline
Changer tous les repliements    &	Shift+F12                   \\ \hline
\caption{Raccourcis de base}
\end{longtable}
}

Ceci est une liste des raccourcis fournis par le composant éditeur de \codeblocks. Ces raccourcis ne peuvent pas être substitués.

{\small
\begin{longtable}{|l|l|}\hline
\textbf{Fonction}		                        &	\textbf{Raccourci clavier}  \\ \hline
\endhead    % Pour répéter la ligne de titre si besoin
Augmenter la taille du texte. 	                &   Ctrl+Keypad "+"             \\ \hline
Diminuer la taille du texte                     &   Ctrl+Keypad "-"             \\ \hline
Restituer la taille normale du texte            &   Ctrl+Keypad "/"             \\ \hline
Permutation circulaire sur les fichiers récents &   Ctrl+Tab                    \\ \hline
Indenter le bloc. 	                            &   Tab                         \\ \hline
Désindenter le bloc.                            &   Shift+Tab                   \\ \hline
Supprimer jusqu'au début du mot.                &   Ctrl+BackSpace              \\ \hline
Supprimer jusqu'à la fin du mot.                &   Ctrl+Delete                 \\ \hline
Supprimer jusqu'au début de ligne.              &   Ctrl+Shift+BackSpace        \\ \hline
Supprimer jusqu'à la fin de ligne.              &   Ctrl+Shift+Delete           \\ \hline
Aller en début de document. 	                &   Ctrl+Home                   \\ \hline
Étendre la sélection jusqu'au début du document.&   Ctrl+Shift+Home             \\ \hline
Aller au début de la ligne affichée.            &   Alt+Home                    \\ \hline
Étendre la sélection jusqu'au début de la ligne.&   Alt+Shift+Home              \\ \hline
Aller à la fin du document. 	                &   Ctrl+End                    \\ \hline
Étendre la sélection jusqu'à la fin du document.&   Ctrl+Shift+End              \\ \hline
Aller à la fin de la ligne affichée             &   Alt+End                     \\ \hline
Étendre la sélection jusqu'à la fin de la ligne.&   Alt+Shift+End               \\ \hline
Étendre ou replier un point de repli. 	        &   Ctrl+Keypad "*"             \\ \hline
Créer ou supprimer un signet	                &	Ctrl+F2                     \\ \hline
Aller au signet suivant		                    &	F2                          \\ \hline
Sélectionner jusqu'au signet suivant            &	Alt+F2                      \\ \hline
Rechercher la sélection.			            & 	Ctrl+F3                     \\ \hline
Rechercher la sélection en arrière.             &	Ctrl+Shift+F3               \\ \hline
Défiler vers le haut. 	                        &   Ctrl+Up                     \\ \hline
Défiler vers le bas. 	                        &   Ctrl+Down                   \\ \hline
Couper la ligne. 	                            &   Ctrl+L                      \\ \hline
Copie de Ligne. 	                            &   Ctrl+Shift+T                \\ \hline
Suppression de ligne. 	                        &   Ctrl+Shift+L                \\ \hline
Permuter la Ligne avec la précédente. 	        &   Ctrl+T                      \\ \hline
Dupliquer la Ligne. 	                        &   Ctrl+D                      \\ \hline
\makecell[l]{Recherche des conditions concordantes du \\
préprocesseur, passer les imbriquées} 	        &   Ctrl+K                      \\ \hline
\makecell[l]{Sélectionner jusqu'aux conditions concordantes du \\
préprocesseur} 	                                &   Ctrl+Shift+K                \\ \hline
\makecell[l]{Recherche des conditions concordantes du \\
préprocesseur en arrière, passer les imbriquées.} 	    &   Ctrl+J              \\ \hline
\makecell[l]{Sélectionner en arrière jusqu'aux conditions \\
concordantes du préprocesseur}	                &   Ctrl+Shift+J                \\ \hline
Paragraphe précédent. Maj étend la sélection.   &   Ctrl+[                      \\ \hline
Paragraphe suivant. Maj étend la sélection.	    &   Ctrl+]                      \\ \hline
Mot précédent. Maj étend la sélection.          &   Ctrl+Left                   \\ \hline
Mot suivant. Maj étend la sélection 	        &   Ctrl+Right                  \\ \hline
Mot partiel précédent. Maj étend la sélection.  &   Ctrl+/                      \\ \hline
Mot partiel suivant. Maj étend la sélection.    &   Ctrl+\osp                   \\ \hline
\caption{Autres Raccourcis de l'éditeur}
\end{longtable}
}

\subsection{Fichiers}

{\small 
\begin{longtable}{|l|l|}\hline
\textbf{Fonction}		                &	\textbf{Raccourci clavier}  \\ \hline
\endhead    % Pour répéter la ligne de titre si besoin
Nouveau fichier ou projet 	            &	Ctrl+N                      \\ \hline
Ouvrir un fichier ou un projet existant &	Ctrl+O                      \\ \hline
Enregistrer le fichier courant 	        &	Ctrl+S                      \\ \hline
Enregistrer tous les fichiers 	        &	Ctrl+Shift+S                \\ \hline
Fermer le fichier courant 	            &	Ctrl+F4/Ctrl+W              \\ \hline
Fermer tous les fichiers 	            &	Ctrl+Shift+F4/Ctrl+Shift+W  \\ \hline
\caption{Raccourcis spécifiques aux fichiers}
\end{longtable}
}

Ceci est une liste des raccourcis fournis par le composant éditeur de \codeblocks. Ces raccourcis ne peuvent pas être substitués.

{\small 
\begin{longtable}{|l|l|}\hline
\textbf{Fonction}		            &	\textbf{Raccourci clavier}  \\ \hline
\endhead    % Pour répéter la ligne de titre si besoin
Activer le fichier ouvert suivant	&   Ctrl+Tab                    \\ \hline
Activer le fichier ouvert précédent &   Ctrl+Shift+Tab              \\ \hline
\caption{Autres raccourcis pour les fichiers}
\end{longtable}
}

\subsection{Vue}

{\small 
\begin{longtable}{|l|l|}\hline
\textbf{Fonction}		                                &	\textbf{Raccourci clavier}  \\ \hline
\endhead    % Pour répéter la ligne de titre si besoin
Afficher / masquer le panneau de Messages	            &	F2                          \\ \hline
Afficher / masquer le panneau de Gestion	            &	Shift+F2                    \\ \hline
Déplacer le projet vers le haut (dans l'arborescence)   &   Ctrl+Shift+Up               \\ \hline
Déplacer le projet vers le bas  (dans l'arborescence)   &   Ctrl+Shift+Down             \\ \hline
Activer le précédent (dans l'arbre des projets)         & 	Alt+F5                      \\ \hline
Activer le suivant   (dans l'arbre des projets)         & 	Alt+F6                      \\ \hline
Zoomer / Dézoomer 	                                    &   Ctrl+ Molette souris        \\ \hline
Focus editor 	                                        &   CTRL+Alt+E                  \\ \hline
\caption{Raccourcis d'affichages}
\end{longtable}
}

\subsection{Recherche}

{\small 
\begin{longtable}{|l|l|}\hline
\textbf{Fonction}		            &	\textbf{Raccourci clavier}  \\ \hline
\endhead    % Pour répéter la ligne de titre si besoin
Rechercher 		                    &	Ctrl+F                      \\ \hline
Rechercher le suivant 	            &	F3                          \\ \hline
Rechercher le précédent 	        &	Shift+F3                    \\ \hline
Rechercher dans les fichiers 	    &	Crtl+Shift+F                \\ \hline
Remplacer 	                        &	Ctrl+R                      \\ \hline
Remplacer dans les fichiers         &	Ctrl+Shift+R                \\ \hline
Aller à la ligne 	                &	Ctrl+G                      \\ \hline
Aller à la ligne changée suivante 	&	Ctrl+F3                     \\ \hline
Aller à la ligne changée précédente	&	Ctrl+Shift+F3               \\ \hline
Aller au fichier 	                &	Alt+G                       \\ \hline
Aller à la fonction	                &	Ctrl+Alt+G                  \\ \hline
Aller à la fonction précédente      &   Ctrl+PgUp                   \\ \hline
Aller à la fonction suivante        &   Ctrl+PgDn                   \\ \hline
Aller à la déclaration              &   Ctrl+Shift+.                \\ \hline
Aller à l'implémentation            &   Ctrl+.                      \\ \hline
Ouvrir le fichier inclus            &   Ctrl+Alt+.                  \\ \hline
\caption{Raccourcis de Recherches}
\end{longtable}
}

\subsection{Générer}

{\small 
\begin{longtable}{|l|l|}\hline
\textbf{Fonction}		    &	\textbf{Raccourci clavier}  \\ \hline
\endhead    % Pour répéter la ligne de titre si besoin
Générer 		            &	Ctrl+F9                     \\ \hline
Compiler le fichier courant	&	Ctrl+Shift+F9               \\ \hline
Exécuter		            &	Ctrl+F10                    \\ \hline
Générer et exécuter 	    &	F9                          \\ \hline
Re-générer 	                &	Ctrl+F11                    \\ \hline
\caption{Raccourcis de Génération}
\end{longtable}
}

\subsection{Debug}

{\small 
\begin{longtable}{|l|l|}\hline
\textbf{Fonction}		                &	\textbf{Raccourci clavier}  \\ \hline
\endhead    % Pour répéter la ligne de titre si besoin
Débuguer 	                            &   F8                          \\ \hline
Continuer le débogage 	                &   Ctrl+F7                     \\ \hline
Aller jusqu'au bloc de code suivant     &   F7                          \\ \hline
Entrer dans le bloc de code	            &   Shift+F7                    \\ \hline
Aller jusqu'en sortie du bloc de code	&   Ctrl+Shift+F7               \\ \hline
Changer l'état du point d'arrêt 	    &   F5                          \\ \hline
Exécuter jusqu'au curseur	            &   F4                          \\ \hline
Erreur précédente	                    &   Alt+F1                      \\ \hline
\caption{Raccourcis du Débugueur}
\end{longtable}
}